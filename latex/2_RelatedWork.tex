%!TEX root = guided_inpainting_paper.tex
\section{Related Work}

\noindent\textbf{Deep Image Generation and Manipulation} Generative Adversarial Network (GAN)~\cite{goodfellow2014generative} uses a mini-max two-player game to alternatively train a generator and a discriminator, and has shown impressive ability to generate natural-looking images of high-quality. However, training instability of the original GAN makes it hard to scale to large images, and many more advanced techniques have been proposed~\cite{denton2015deep,radford2015unsupervised,zhao2016energy,arjovsky2017wasserstein,gulrajani2017improved}. Recently, Progressive GAN~\cite{karras2017progressive} is proposed which can be trained to generate images of unprecedented quality. Our procedural block-wise training and Progressive GAN share the basic idea of progressively increasing the depth of the network while training. However, both the problems being studied (image generation vs conditional synthesis) and the model architectures used are different. 

Adversarial training has also been applied to many image editing tasks~\cite{kim2016accurate,dong2014learning,ledig2016photo,isola2016image,zhu2017unpaired}. For image inpainting, many DNN-based approaches achieve good performance using different network topology and training procedure ~\cite{pathak2016context,yang2017high,yeh2016semantic,iizuka2017globally}. For image harmonization which aims to adjust the appearances of image composition such that it looks more natural and plausible, recent approaches mostly use deep neural network to leverage its expressive power and semantic knowledge~\cite{zhu2015learning,tsai2017deep}.

\noindent\textbf{Non-neural Inpainting and Harmonization} Traditional image completion algorithms can be either diffusion-based~\cite{bertalmio2000image,elad2005simultaneous} or patch-based~\cite{bertalmio2003simultaneous,barnes2009patchmatch}. Diffusion-based methods usually cannot synthesize plausible contents for large holes or textures, due to the fact that it only propagates low-level features. Patch-based methods, however, largely rely on the assumption that the desired patches exist in the database. For harmonization, traditional methods usually apply color and tone optimization, by matching global or multi-scale statistics~\cite{reinhard2001color,sunkavalli2010multi}, extracting gradient domain information~\cite{perez2003poisson,tao2010error}, utilizing semantic clues~\cite{tsai2016sky} or leveraging external data~\cite{johnson2011cg2real}. 
